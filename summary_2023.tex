\section*{索状態ロボット特長洗い出し}	
	\begin{itemize}
	\item{アプローチ}
		\begin{itemize}
			\item {索状体ユニットの組み合わせで様々な不整地環境を踏破する}
			%\item 天井裏を継続するかどうか
		\end{itemize}
	\item {車輪と比較した索状態ユニットの利点}
	\begin{itemize}
		\item {駆動部が露出しにくい}
		\item {柔らかい外皮で覆うことができる}
		\item {地面との接触面積が大きい}
		\item {底面における土質のせん断破壊の軌道が進行方向に対して平行かつ逆向き\\
		(クローラ特性)→クローラと車輪では牽引力はクローラの方が大きいことが示されている.\\
		\lbrack 文献\rbrack Wheel vs Tracks -A Fundamental...}
	\end{itemize}
	\item{クローラに比較した索状態ユニットの利点}
	\begin{itemize}
		\item {駆動部が露出しない}
		\item {草や小石が駆動部に入りにくい}
		\item {水上や水中での移動可能性}
		\item {高さが低い環境でも動くことができる\\
		クローラーは底面と上面が外部環境と面接触していると進行できない}
	\end{itemize}
	\item{ヘビ型ロボットと比較}
	\begin{itemize}
		\item {アクチュエータの構成を簡単化できる?(サーボからDC)}
		\item {1つのモータで前進後退できる}
	\end{itemize}
	\end{itemize}
	%\item {NIF}           
\newpage

\section*{研究大目的}
	\begin{itemize}
	\item {環境に対応して組換可能な単モータ型索状体ロボットの開発}
	%\item{前進運動の構築}
	%\item{車輪型アタッチメントパーツについて}
	\end{itemize}
	
\section*{今回の研究背景}
\begin{itemize}
	\item{超冗長ロボットの導入が盛んになってきているが...}
	\begin{itemize}
		\item {金額コストや電力コストなど導入への障壁が大きい}
		\item {ケーブルラックや天井裏など様々な環境を1台で踏破できる性能は魅力}
		\item {冗長ロボットの中でも水陸で動作実績があるヘビ型ロボットについて考える}
		\item {ヘビ型ロボットの特徴の進行波推進を残したままアクチュエータや電力,速度コストを下げる}
		\item {レゴのように組み換え可能なら自由度設計などシステムアーキテクチャ的に求められる環境,
		動作に対して最良の構成をすることができる.\\
		冗長ロボットの場合は高い自由度を確保して様々な環境に対応するため,コスト面での調整が難しい?}
		%\item{機械学習}
	\end{itemize}
\end{itemize}

\section*{今後の予定}
	\begin{itemize}
		\item{論文調査}
		\item {乗り越え時のユニットねじれ解消}
		\item {直列機構での野縁受け乗り越え}
		\item {並列機構をつないで障害物乗り越え}
		%%\item {ケーブルラック上移動}
	\end{itemize}